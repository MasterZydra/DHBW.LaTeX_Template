\documentclass[
fontsize=12pt,      %% size of the main text
parskip=\myparskip, %% vertical space between paragraphs (instead of indenting first par-line)
DIV=calc,           %% calculates a good DIV value for type area; 66 characters/line is great
headinclude=true,   %% is header part of margin space or part of page content?
footinclude=false,  %% is footer part of margin space or part of page content?
open=right,         %% "right" or "left": start new chapter on right or left page
appendixprefix=true,%% adds appendix prefix; only for book-classes with \backmatter
bibliography=totoc, %% adds the bibliography to table of contents (without number)
BCOR=\myBCOR,       %% binding correction (depends on how you bind the resulting printout.
\mylaterality       %% oneside: document is not printed on left and right sides, only right side
                    %% twoside: document is printed on left and right sides
]{scrbook}

%% Set paper and border size
\usepackage[paper=a4paper,left=25mm,right=25mm,top=25mm,bottom=25mm]{geometry}

%% UTF8 as input characters; UCS incompatible to biblatex
\usepackage[utf8]{inputenc}

%% The default setting of the language is American. Please change settings for
%% additional or alternative languages used in main.tex.
%% Please note that the default language of the document is the *last* language
%% which is added to the package options.
%% To set only parts of your document in a different language as the rest,
%% use for example\newline\verb+\foreignlanguage{ngerman}{Beispieltext in deutscher Sprache}+\newline
\usepackage[\mylanguage]{babel}

%% This package defines basic colors. If you want to get rid of colored links
%% and headings please change corresponding value in main.tex to {0,0,0}.
%% Used for links and so forth in screen-version
\usepackage[usenames,dvipsnames]{xcolor}
\definecolor{DispositionColor}{RGB}{\mydispositioncolor}

%% The widely used package to use graphical images within a LaTeX document.
%% \includegraphics[width=42mm]{figures/image}
\usepackage[pdftex]{graphicx}

%% For example on title pages you might want to have a logo on the upper right
%% corner of the first page (only). The package \texttt{eso-pic} is able to
%% place things on absolute and relative positions on the whole page.
\usepackage{eso-pic}

%% List of abbreviations
\usepackage{acronym}

%% Use biblatex for bibliography
%% With "Sorting=None" the numbering is done according to the order of appearance. 
\usepackage[style=numeric,sorting=none]{biblatex}

%% Used for quotes
\usepackage{csquotes}

%% Add fontec to hyphenate umlaut and fix-cm to fix changes in section heading
\usepackage[T1]{fontenc}
\usepackage{fix-cm}

%% Add default bibliography
\addbibresource{etc/literature.bib}

%% Give name to bibliography
\bibliography{Literaturverzeichnis}

%% ========================================================================
%%%% Typographic settings
%% ========================================================================

%% If you have to enlarge the distance between two lines of text, you can
%% increase it using the \texttt{\mylinespread} command in \texttt{main.tex}. By default, it is
%% deactivated (set to 100~percent). Modify only with caution since it influences the
%% page layout and could lead to ugly looking documents.
\linespread{\mylinespread}

\renewcommand*{\chapterheadstartvskip}{\vspace*{0\baselineskip}}% Abstand einstellen

%% Reduce space between section and chapter headings
\RedeclareSectionCommand[
   beforeskip=0pt,
   afterskip=1sp
]{chapter}

\RedeclareSectionCommand[
   beforeskip=0pt,
   afterskip=1sp
]{section}

\RedeclareSectionCommand[
   beforeskip=0pt,
   afterskip=1sp
]{subsection}

\RedeclareSectionCommand[
   beforeskip=0pt,
   afterskip=1sp
]{subsubsection}

%% This document template is able to generate an output that uses colorized
%% headings, captions, page numbers, and links. The color named
%% `DispositionColor' used in this document is defined near the definition
%% of package xcolor
%% The changes required for headings, page numbers and captions are defined
%% here.
%% Settings for colored links are handled by the definitions of the
%% hyperref package
%% \setheadsepline{.4pt}[\color{DispositionColor}]
\renewcommand{\headfont}{\normalfont\sffamily\color{DispositionColor}}
\renewcommand{\pnumfont}{\normalfont\sffamily\color{DispositionColor}}
\addtokomafont{disposition}{\color{DispositionColor}}
\addtokomafont{caption}{\color{DispositionColor}\footnotesize}
\addtokomafont{captionlabel}{\color{DispositionColor}}


%% Add package to add long text tables
\usepackage{longtable}

%% ========================================================================
%%%% Source code highlighting
%% ========================================================================

%% Minted for source code highlighting
\usepackage[newfloat]{minted}

%%\BeforeBeginEnvironment{minted}{\medskip}
%%\AfterEndEnvironment{minted}{\smallskip}

%% XColor for color definitions
\usepackage{xcolor}

%% Define background color for style "friendly"
\definecolor{friendlybg}{HTML}{f0f0f0}

%% Settings for source code
\definecolor{codebg}{rgb}{0.95,0.95,0.95}
\setminted{
    bgcolor=codebg,
    startinline=true,
    obeytabs=true,
    tabsize=4,
    linenos,
    breaklines,
    fontsize=\small,
    baselinestretch=1.2,
    samepage=true
}

\renewcommand{\listingname}{Auflistung}


\usepackage{etoolbox}

\makeatletter
\AtBeginEnvironment{minted}{\dontdofcolorbox}
\def\dontdofcolorbox{\renewcommand\fcolorbox[4][]{##4}}
\makeatother
