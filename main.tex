%% ========================================================================
%%%% Document settings
%% ========================================================================

%% How to separate paragraphs: indention ("no") or spacing ("half", "full",
%% ...).
\newcommand{\myparskip}{half}

%% Inner binding correction. This value depends on the method which is
%% being used to bind your printed result. Some techniques do not
%% require a binding correction at all ("0mm"), other require for
%% example "5mm". Refer to KOMA script documentation for a detailed
%% explanation what a binding correction is and how to measure it.
\newcommand{\myBCOR}{0mm}

%% Either you are creating a document which is printed on both, left pages
%% and right pages (twoside) or you create a document which is printed
%% on right pages only (oneside).
%% e.g. "oneside" or "twoside"
\newcommand{\mylaterality}{oneside}

%% Color of the headings and so forth in RGB (red,green,blue) values.
%% e.g. "30,103,182" (blue/turquois), "0,0,0" (black), ...
\newcommand{\mydispositioncolor}{22,72,154}

%% Line spacing in %/100. For example 1.5 means 150% of the usual line
%% spacing. Please use with caution: 100% ("1.0") is fine because the
%% font was designed for it.
%% e.g. 1.0, 1.5, 2.0
\newcommand{\mylinespread}{1.5}

\newcommand{\mycolorlinks}{true}  %% "true" or "false"
%% Enables or disables colored links (hyperref package).
% turn on/off colored links (on: better for
                          % on-screen reading; off: better for printout versions)

%% See babel documentation for further details.
%% NOTE: The *last* language is the active one!
%% "english,ngerman", "ngerman,english", ...
\newcommand{\mylanguage}{american,ngerman}

%% ========================================================================
%%%% Document metadata
%% ========================================================================

%% Student:
\newcommand{\myauthor}{Max Mustermann}
\newcommand{\myhomestreet}{Musterstraße 1}
\newcommand{\myhomepostalnumber}{12345}
\newcommand{\myhometown}{Musterhausen}
\newcommand{\myid}{123456789}

%% Titel und PDF-Einstellungen
\newcommand{\myworktype}{Projektarbeit}
\newcommand{\myworktitle}{T3\_1000}
\newcommand{\mytitle}{Titel einer sehr komplexen Arbeit}
\newcommand{\mysubtitle}{ }
\newcommand{\mysubject}{T3\_1000 von Max Mustermann}
\newcommand{\mykeywords}{Key, Word}

%% Dualer Partner
\newcommand{\mydualpartner}{Ein Unternehmen AG}
\newcommand{\mydualpartnerLocation}{12345 Unternehmertum}
\newcommand{\mysupervisor}{Berthold Betreuer}

%% Duale Hochschule
\newcommand{\myuniversity}{Duale Hochschule Muster}
\newcommand{\mycourse}{MUS-INFORMATIK}
\newcommand{\mystudy}{Informatik}
\newcommand{\myevaluator}{Prof. Dr. Dr. Müller}

%% Datum
\newcommand{\mysubmissionday}{1}
\newcommand{\mysubmissionmonth}{April}
\newcommand{\mysubmissionyear}{2020}

\newcommand{\myprocessingperiodbegin}{01.01.2020}
\newcommand{\myprocessingperiodend}{01.04.2020}

%% ========================================================================
%%%% Other settings
%% ========================================================================

%% !!FIRST!! Load preamble
\documentclass[
fontsize=12pt,      %% size of the main text
parskip=\myparskip, %% vertical space between paragraphs (instead of indenting first par-line)
DIV=calc,           %% calculates a good DIV value for type area; 66 characters/line is great
headinclude=true,   %% is header part of margin space or part of page content?
footinclude=false,  %% is footer part of margin space or part of page content?
open=right,         %% "right" or "left": start new chapter on right or left page
appendixprefix=true,%% adds appendix prefix; only for book-classes with \backmatter
bibliography=totoc, %% adds the bibliography to table of contents (without number)
BCOR=\myBCOR,       %% binding correction (depends on how you bind the resulting printout.
\mylaterality       %% oneside: document is not printed on left and right sides, only right side
                    %% twoside: document is printed on left and right sides
]{scrbook}

%% Set paper and border size
\usepackage[paper=a4paper,left=25mm,right=25mm,top=25mm,bottom=25mm]{geometry}

%% UTF8 as input characters; UCS incompatible to biblatex
\usepackage[utf8]{inputenc}

%% The default setting of the language is American. Please change settings for
%% additional or alternative languages used in main.tex.
%% Please note that the default language of the document is the *last* language
%% which is added to the package options.
%% To set only parts of your document in a different language as the rest,
%% use for example\newline\verb+\foreignlanguage{ngerman}{Beispieltext in deutscher Sprache}+\newline
\usepackage[\mylanguage]{babel}

%% This package defines basic colors. If you want to get rid of colored links
%% and headings please change corresponding value in main.tex to {0,0,0}.
%% Used for links and so forth in screen-version
\usepackage[usenames,dvipsnames]{xcolor}
\definecolor{DispositionColor}{RGB}{\mydispositioncolor}

%% The widely used package to use graphical images within a LaTeX document.
%% \includegraphics[width=42mm]{figures/image}
\usepackage[pdftex]{graphicx}

%% For example on title pages you might want to have a logo on the upper right
%% corner of the first page (only). The package \texttt{eso-pic} is able to
%% place things on absolute and relative positions on the whole page.
\usepackage{eso-pic}

%% List of abbreviations
\usepackage{acronym}

%% Use biblatex for bibliography
%% With "Sorting=None" the numbering is done according to the order of appearance. 
\usepackage[style=numeric,sorting=none]{biblatex}

%% Used for quotes
\usepackage{csquotes}

%% Add fontec to hyphenate umlaut and fix-cm to fix changes in section heading
\usepackage[T1]{fontenc}
\usepackage{fix-cm}

%% Add default bibliography
\addbibresource{etc/literature.bib}

%% Give name to bibliography
\bibliography{Literaturverzeichnis}

%% ========================================================================
%%%% Typographic settings
%% ========================================================================

%% If you have to enlarge the distance between two lines of text, you can
%% increase it using the \texttt{\mylinespread} command in \texttt{main.tex}. By default, it is
%% deactivated (set to 100~percent). Modify only with caution since it influences the
%% page layout and could lead to ugly looking documents.
\linespread{\mylinespread}

\renewcommand*{\chapterheadstartvskip}{\vspace*{0\baselineskip}}% Abstand einstellen

%% Reduce space between section and chapter headings
\RedeclareSectionCommand[
   beforeskip=0pt,
   afterskip=1sp
]{chapter}

\RedeclareSectionCommand[
   beforeskip=0pt,
   afterskip=1sp
]{section}

\RedeclareSectionCommand[
   beforeskip=0pt,
   afterskip=1sp
]{subsection}

\RedeclareSectionCommand[
   beforeskip=0pt,
   afterskip=1sp
]{subsubsection}

%% This document template is able to generate an output that uses colorized
%% headings, captions, page numbers, and links. The color named
%% `DispositionColor' used in this document is defined near the definition
%% of package xcolor
%% The changes required for headings, page numbers and captions are defined
%% here.
%% Settings for colored links are handled by the definitions of the
%% hyperref package
%% \setheadsepline{.4pt}[\color{DispositionColor}]
\renewcommand{\headfont}{\normalfont\sffamily\color{DispositionColor}}
\renewcommand{\pnumfont}{\normalfont\sffamily\color{DispositionColor}}
\addtokomafont{disposition}{\color{DispositionColor}}
\addtokomafont{caption}{\color{DispositionColor}\footnotesize}
\addtokomafont{captionlabel}{\color{DispositionColor}}


%% Add package to add long text tables
\usepackage{longtable}

%% ========================================================================
%%%% Source code highlighting
%% ========================================================================

%% Minted for source code highlighting
\usepackage[newfloat]{minted}

%%\BeforeBeginEnvironment{minted}{\medskip}
%%\AfterEndEnvironment{minted}{\smallskip}

%% XColor for color definitions
\usepackage{xcolor}

%% Define background color for style "friendly"
\definecolor{friendlybg}{HTML}{f0f0f0}

%% Settings for source code
\definecolor{codebg}{rgb}{0.95,0.95,0.95}
\setminted{
    bgcolor=codebg,
    startinline=true,
    obeytabs=true,
    tabsize=4,
    linenos,
    breaklines,
    fontsize=\small,
    baselinestretch=1.2,
    samepage=true
}

\renewcommand{\listingname}{Auflistung}


\usepackage{etoolbox}

\makeatletter
\AtBeginEnvironment{minted}{\dontdofcolorbox}
\def\dontdofcolorbox{\renewcommand\fcolorbox[4][]{##4}}
\makeatother

%% for the code-block environment with listing and reference possibility:
%% \begin{code}{<language>}{<caption>}{<label>} 
%% ... 
%% \end{code}
%% is also used with listings so there is a list of code blocks
\usepackage{listings}
\usepackage[lstlisting]{totalcount}
\renewcommand{\lstlistingname}{Code}
\usepackage{caption}
\captionsetup[lstlisting]{skip=5pt,belowskip=15pt} %spacing for the label
%% the following see also here: https://github.com/gpoore/minted/issues/108
\newcommand{\storeCaption}{}
\newcommand{\storeLabel}{}
\newenvironment{codelisting}{\captionsetup{type=lstlisting}}{}
\newenvironment{code}[3]
{%
	\VerbatimEnvironment
	\renewcommand{\storeCaption}{#2}%
	\renewcommand{\storeLabel}{#3}%
	\begin{minipage}{\linewidth}%
	\begin{codelisting}%
	\begin{minted}{#1}%
	}
	{%
	\end{minted}%
	\caption{\storeCaption}%
	\label{\storeLabel}%
	\end{codelisting}%
	\end{minipage}\\%
}


\hyphenation{Get-Stream-With-Image-Rotated-For-External-Storage ab-hör-si-cher-en Funk-ti-ons-um-fang Dev-EUI Bild-er-ken-nung}

%% !!LAST!! Do PDF settings
%% ========================================================================
%%%% PDF settings
%% ========================================================================

\pdfcompresslevel=9

%% Declarations of hyperref should be the last definitions of the preamble
\usepackage[
unicode=true,
backref,
pagebackref=false,
bookmarks=true,
bookmarksopen=false,
pdfpagemode=UseNone,
plainpages=false,
urlcolor=DispositionColor,
linkcolor=DispositionColor,
citecolor=DispositionColor,
anchorcolor=DispositionColor,
colorlinks=\mycolorlinks,
]{hyperref}

%% all strings need to be loaded after hyperref was loaded with unicode support
%% if not the field is garbled in the output for characters like ČŽĆŠĐ
\hypersetup{
pdftitle = {\mytitle},
pdfauthor = {\myauthor},
pdfsubject = {\mysubject},
pdfproducer = {\myauthor},
pdfkeywords = {\mykeywords}
}

%% ========================================================================
%%%% begin{document}
%% ========================================================================
\begin{document}
\sffamily
\frontmatter

%% Titelseite
%% Layout mit Firmen und DHBW Angaben
%% ========================================================================
%%%% Information
%% ========================================================================
%% Titelseite für Arbeiten mit der dualen Hochschule Mosbach und dem
%% Unternehmen WIKA.
%% Die anzuzeigenden Daten werden in der Datei main.tex festgelegt.

\begin{titlepage}

%% WIKA Logo
\AddToShipoutPicture*{
    \AtPageUpperLeft{
        \hspace{\paperwidth}
        \raisebox{-50mm}{
            \makebox[-129mm][r]{\includegraphics[width=42mm]{template/figures/Company_Logo}}
        }
    }
}

%% DHBW Logo
\AddToShipoutPicture*{
    \AtPageUpperLeft{
        \hspace{\paperwidth}
        \raisebox{-52mm}{
            \makebox[-39mm][r]{\includegraphics[width=42mm]{template/figures/University_Logo}}
        }
    }
}

\begin{center}
~
\vfill\vfill\vfill \vfill

\sffamily

{\LARGE\bfseries\mytitle}

{\large\bfseries\mysubtitle}

\vfill\vfill
{\normalsize\bfseries\myworktype~\myworktitle}\\

\vfill \vfill\vfill
Im Studiengang \mystudy ~an der\\
{\normalsize\bfseries\myuniversity}

\vfill
von\\
\myauthor

\vfill
\mysubmissionday. \mysubmissionmonth~\mysubmissionyear

\vfill\vfill\vfill
\end{center}

\sffamily{
\begin{tabular}{ll}
    Bearbeitungszeitraum & \myprocessingperiodbegin ~- \myprocessingperiodend \\
    Matrikelnummer & \myid \\
    Kurs & \mycourse \\
    Ausbildungsfirma & \mydualpartner \\
    & \mydualpartnerLocation \\
    Betreuer der Ausbildungsfirma & \mysupervisor \\
    Gutachter der Dualen Hochschule & \myevaluator \\
\end{tabular}
}

\end{titlepage}

\newpage
%% Verwendung von Projekt-Deckblatt
%% %% ========================================================================
%%%% Information
%% ========================================================================
%% Titelseite für Arbeiten mit der dualen Hochschule Mosbach und dem
%% Unternehmen WIKA.
%% Die anzuzeigenden Daten werden in der Datei main.tex festgelegt.

\begin{titlepage}

%% Company Logo
\AddToShipoutPicture*{
    \AtPageUpperLeft{
        \hspace{\paperwidth}
        \raisebox{-50mm}{
            \makebox[-129mm][r]{\includegraphics[width=42mm]{template/figures/DHBW.png}}
        }
    }
}

%% University Logo
\AddToShipoutPicture*{
    \AtPageUpperLeft{
        \hspace{\paperwidth}
        \raisebox{-52mm}{
            \makebox[-39mm][r]{\includegraphics[width=42mm]{template/figures/Company_Logo.png}}
        }
    }
}

\begin{center}
~
\vfill\vfill\vfill \vfill\vfill

\sffamily

{\LARGE\bfseries\mytitle}

{\large\bfseries\mysubtitle}

\vfill\vfill
{\normalsize\bfseries\myworktype~\myworktitle}\\

\vfill \vfill
Im Studiengang \mystudy\\
an der \myuniversity

\vfill
von\\
\textbf{\myauthor}

\vfill
\mysubmissionday. \mysubmissionmonth~\mysubmissionyear

\vfill\vfill\vfill
\end{center}

\sffamily{
\begin{tabular}{ll}
    Matrikelnummer & \myid \\
    Ausbildungsfirma & \mydualpartner \\
    & \mydualpartnerLocation \\
    Betreuer der Ausbildungsfirma & \mysupervisor \\
\end{tabular}
}

\end{titlepage}

\newpage

%% Verwendung von Projekt-Deckblatt
%% %% ========================================================================
%%%% Information
%% ========================================================================
%% Titelseite für Arbeiten mit der dualen Hochschule Mosbach und dem
%% Unternehmen WIKA.
%% Die anzuzeigenden Daten werden in der Datei main.tex festgelegt.

\begin{titlepage}

%% Company Logo
\AddToShipoutPicture*{
    \AtPageUpperLeft{
        \hspace{\paperwidth}
        \raisebox{-50mm}{
            \makebox[-129mm][r]{\includegraphics[width=42mm]{template/figures/Company_Logo.png}}
        }
    }
}

%% University Logo
\AddToShipoutPicture*{
    \AtPageUpperLeft{
        \hspace{\paperwidth}
        \raisebox{-52mm}{
            \makebox[-39mm][r]{\includegraphics[width=42mm]{template/figures/DHBW.png}}
        }
    }
}

\begin{center}
~
\vfill\vfill\vfill \vfill\vfill

\sffamily

{\LARGE\bfseries\mytitle}

{\large\bfseries\mysubtitle}

\vfill\vfill
{\normalsize\bfseries\ BACHELORARBEIT}\\

\vfill\vfill
für die Prüfung zum \\
\mydegree

\vfill \vfill
des Studiengangs \mystudy\\
an der \myuniversity

\vfill
von\\
\textbf{\myauthor}

\vfill
\mysubmissionday. \mysubmissionmonth~\mysubmissionyear

\vfill\vfill\vfill
\end{center}

\sffamily{
\begin{tabular}{ll}
    Bearbeitungszeitraum & 12 Wochen \\
    Matrikelnummer, Kurs & \myid, \mycourse \\
    Ausbildungsfirma & \mydualpartner, \mydualpartnerLocation \\
    Betreuer der Ausbildungsfirma & \mysupervisor \\
    Gutachter der Dualen Hochschule & \myevaluator
\end{tabular}
}

\end{titlepage}

\newpage


%% Sperrvermerk
%% Sperrvermerk
\section*{Sperrvermerk}
Die vorgelegte \myworktype ~mit dem Titel "\textit{{\mytitle}}" beinhaltet vertrauliche Informationen und Daten des Unternehmens \mydualpartner.

Diese \myworktype ~darf nur vom Erst- und Zweitgutachter sowie berechtigten Mitgliedern des Prüfungsausschusses eingesehen werden. Eine Vervielfältigung und Veröffentlichung der \myworktype ~ist auch auszugsweise nicht erlaubt.

Dritten darf diese Arbeit nur mit der ausdrücklichen Genehmigung des Verfassers und Unternehmens zugänglich gemacht werden.

\newpage
%% Eidesstattliche Erklärung
%% Neuen Befehl für Textfelder definieren
\newcommand{\textfield}[2]{
    \vbox{
        \hsize=#1\kern3cm\hrule\kern1ex
        \hbox to \hsize{\strut\hfil\footnotesize#2\hfil}
    }
}

%% Eidesstattliche Erklärung
\section*{Erklärung}
Ich versichere hiermit, dass ich meine \myworktype ~mit dem Thema: "\textit{{\mytitle}}" selbstständig verfasst und keine anderen als die angegebenen Quellen und Hilfsmittel benutzt habe.
Ich versichere zudem, dass die eingereichte elektronische Fassung mit der gedruckten Fassung übereinstimmt. \\
%% Ort, Datum und Unterschrift
\hbox to \hsize{\textfield{7cm}{Ort, Datum}\hfil\hfil\textfield{5cm}{Unterschrift}}

\newpage


%% Abstract
%% ---------------------------------------------
%% Kurzfassung
%% ---------------------------------------------
\chapter*{Kurzfassung} \label{chap:Kurzfassung}
Lorem ipsum dolor sit amet, consetetur sadipscing elitr, sed diam nonumy eirmod tempor invidunt ut labore et dolore magna aliquyam erat, sed diam voluptua. At vero eos et accusam et justo duo dolores et ea rebum. Stet clita kasd gubergren, no sea takimata sanctus est Lorem ipsum dolor sit amet. Lorem ipsum dolor sit amet, consetetur sadipscing elitr, sed diam nonumy eirmod tempor invidunt ut labore et dolore magna aliquyam erat, sed diam voluptua. At vero eos et accusam et justo duo dolores et ea rebum. Stet clita kasd gubergren, no sea takimata sanctus est Lorem ipsum dolor sit amet.

%% ---------------------------------------------
%% Abstract
%% ---------------------------------------------
\begingroup
% Group for removing the page break
\renewcommand{\cleardoublepage}{}
\renewcommand{\clearpage}{}
\chapter*{Abstract} \label{chap:abstract}
\endgroup
Lorem ipsum dolor sit amet, consetetur sadipscing elitr, sed diam nonumy eirmod tempor invidunt ut labore et dolore magna aliquyam erat, sed diam voluptua. At vero eos et accusam et justo duo dolores et ea rebum. Stet clita kasd gubergren, no sea takimata sanctus est Lorem ipsum dolor sit amet. Lorem ipsum dolor sit amet, consetetur sadipscing elitr, sed diam nonumy eirmod tempor invidunt ut labore et dolore magna aliquyam erat, sed diam voluptua. At vero eos et accusam et justo duo dolores et ea rebum. Stet clita kasd gubergren, no sea takimata sanctus est Lorem ipsum dolor sit amet.

%% Inhaltsverzeichnis
\tableofcontents

%% Abkürzungen
\chapter{Abkürzungsverzeichnis}
\begin{acronym}
    \acro{Abkürzung}{Beschreibung der Abkürzung}
\end{acronym}

%% Abbildungsverzeichnis
\addtocontents{lof}{\protect\addcontentsline{toc}{chapter}{\listfigurename}}
\listoffigures

%% Tabellenverzeichnis
\listoftables

%% Code-Verzeichnis wenn Code-Blöcke irgendwo auftauchen
\renewcommand\lstlistlistingname{Code-Verzeichnis} %Code-Verzeichnis
\iftotallstlistings
    \lstlistoflistings
\fi

%% Marks main part using arabic page numbers and such; Only available in scrbook
\mainmatter

%% Add new chapters here
\chapter{Einleitung}

%% ---------------------------------------------
%% Motivation
%% ---------------------------------------------
\section{Motivation}
Lorem ipsum dolor sit amet, consetetur sadipscing elitr, sed diam nonumy eirmod tempor invidunt ut labore et dolore magna aliquyam erat, sed diam voluptua. At vero eos et accusam et justo duo dolores et ea rebum. Stet clita kasd gubergren, no sea takimata sanctus est Lorem ipsum dolor sit amet. Lorem ipsum dolor sit amet, consetetur sadipscing elitr, sed diam nonumy eirmod tempor invidunt ut labore et dolore magna aliquyam erat, sed diam voluptua.

%% ---------------------------------------------
%% Stand der Technik
%% ---------------------------------------------
\section{Stand der Technik}
Lorem ipsum dolor sit amet, consetetur sadipscing elitr, sed diam nonumy eirmod tempor invidunt ut labore et dolore magna aliquyam erat, sed diam voluptua. At vero eos et accusam et justo duo dolores et ea rebum. Stet clita kasd gubergren, no sea takimata sanctus est Lorem ipsum dolor sit amet. Lorem ipsum dolor sit amet, consetetur sadipscing elitr, sed diam nonumy eirmod tempor invidunt ut labore et dolore magna aliquyam erat, sed diam voluptua.

%% ---------------------------------------------
%% Zielsetzung
%% ---------------------------------------------
\section{Zielsetzung}
Lorem ipsum dolor sit amet, consetetur sadipscing elitr, sed diam nonumy eirmod tempor invidunt ut labore et dolore magna aliquyam erat, sed diam voluptua. At vero eos et accusam et justo duo dolores et ea rebum. Stet clita kasd gubergren, no sea takimata sanctus est Lorem ipsum dolor sit amet. Lorem ipsum dolor sit amet, consetetur sadipscing elitr, sed diam nonumy eirmod tempor invidunt ut labore et dolore magna aliquyam erat, sed diam voluptua.

%% ---------------------------------------------
%% Aufbau der Projektarbeit
%% ---------------------------------------------
\section{Aufbau der \myworktype}
Lorem ipsum dolor sit amet, consetetur sadipscing elitr, sed diam nonumy eirmod tempor invidunt ut labore et dolore magna aliquyam erat, sed diam voluptua. At vero eos et accusam et justo duo dolores et ea rebum. Stet clita kasd gubergren, no sea takimata sanctus est Lorem ipsum dolor sit amet. Lorem ipsum dolor sit amet, consetetur sadipscing elitr, sed diam nonumy eirmod tempor invidunt ut labore et dolore magna aliquyam erat, sed diam voluptua.

\chapter{Beispiele}

%% ---------------------------------------------
%% Abbildung
%% ---------------------------------------------
\section{Abbildung}
In der Abbildung \ref{fig_example} ...

\begin{figure}[!h]
    \centering
    \includegraphics[width=\textwidth]{img/example_image.png}
    \caption{Ein wundervolles Bild in ganzer Breite}
    \label{fig_example}
\end{figure}

Die Abbildung \ref{fig_example_halfsize} hat die 50 Prozent Breite ...

\begin{figure}[h!]
    \centering
    \includegraphics[width=0.5\textwidth]{img/example_image.png}
    \caption[Kurzbeschreibung für Abbildungsverzeichnis]{Ein wundervolles Bild in 50\% Breite}
    \label{fig_example_halfsize}
\end{figure}

%% ---------------------------------------------
%% Abkürzung
%% ---------------------------------------------
\section{Abkürzung}
Eine Verlinkung zur \acs{Abkürzung} ist auch möglich. Jedoch gibt es bei Umlauten Probleme...

%% ---------------------------------------------
%% Anführungszeichen
%% ---------------------------------------------
\section{Anführungszeichen}
\glqq Anführungszeichen\grqq{} können mit \mintinline{latex}{\glqq} für das linke und \mintinline{latex}{\grqq{}} für das rechte Anführungszeichen ergänzt werden.

%% ---------------------------------------------
%% Auflistung
%% ---------------------------------------------
\section{Auflistung}
In der folgenden Auflistung...
\begin{itemize}
    \setlength\itemsep{-1em}
    \item erstes Stichwort
    \item zweites Stichwort
    \item dritte Stichwort
\end{itemize}

Eine Auflistung mit einem anderen Abstand zwischen den Punkten...
\begin{itemize}
    \setlength\itemsep{-0.5em}
    \item erstes Stichwort
    \item zweites Stichwort
    \item dritte Stichwort
\end{itemize}

%% ---------------------------------------------
%% Kurze Tabelle
%% ---------------------------------------------
\section{Kurze Tabelle}
In der Tabelle \ref{tbl_example_table} ...

\begin{table}[!h]
    \centering
    \begin{tabular}{l|ll}
        Datensatz 1     & Wert 1    & Einheit \\
        Datensatz 2     & Wert 2    & Einheit \\ \hline
        $\sum$          & Summe     & Einheit \\
    \end{tabular}
    \caption{Beispiel für eine Tabelle}
    \label{tbl_example_table}
\end{table}

%% ---------------------------------------------
%% Lange Tabelle
%% ---------------------------------------------
\section{Lange Tabelle}
Eine lange Tabelle mit einem alternativen Text für das Tabellenverzeichnis...
\begin{longtable}{p{0.5\linewidth} p{0.5\linewidth}}
    \hline
    Lorem ipsum dolor sit amet, consetetur sadipscing elitr, sed diam nonumy eirmod tempor invidunt ut labore et dolore magna aliquyam erat, sed diam voluptua. &
    Lorem ipsum dolor sit amet, consetetur sadipscing elitr, sed diam nonumy eirmod tempor invidunt ut labore et dolore magna aliquyam erat, sed diam voluptua. \\
    \hline
    Lorem ipsum dolor sit amet, consetetur sadipscing elitr, sed diam nonumy eirmod tempor invidunt ut labore et dolore magna aliquyam erat, sed diam voluptua. &
    Lorem ipsum dolor sit amet, consetetur sadipscing elitr, sed diam nonumy eirmod tempor invidunt ut labore et dolore magna aliquyam erat, sed diam voluptua. \\
    \hline
    Lorem ipsum dolor sit amet, consetetur sadipscing elitr, sed diam nonumy eirmod tempor invidunt ut labore et dolore magna aliquyam erat, sed diam voluptua. &
    Lorem ipsum dolor sit amet, consetetur sadipscing elitr, sed diam nonumy eirmod tempor invidunt ut labore et dolore magna aliquyam erat, sed diam voluptua. \\
    \hline
\caption[Alternativer kürzerer Text für eine Caption]{Hier eine lange Tabelle mit zwei Spalten (je 50\% Breite). Diese könnte auch über mehrere Seiten gehen}
\label{tbl_longtable}
\end{longtable}

%% ---------------------------------------------
%% PDF
%% ---------------------------------------------
\section{PDF}
Mit dem Compiler \textit{pdfLaTeX} ist es möglich andere PDF-Dokumente wie ein Bild einzubinden (siehe \ref{fig_example_pdf}).

\begin{figure}[h!]
    \centering
    \includegraphics[width=0.5\textwidth]{img/Example.pdf}
    \caption[Beispiel für PDF-Einbindung]{Einbinden eines PDFs in LaTeX}
    \label{fig_example_pdf}
\end{figure}

%% ---------------------------------------------
%% Quellcode
%% ---------------------------------------------
\section{Quellcode}
Quellcode kann als Block oder im Fließtext stehen. Der Ausdruck \mintinline{python}{print(x**2)} ist ein Beispiel für Inline-Quelltext.

Folgender Code-Block taucht nicht im Code-Verzeichnis auf:
\begin{minted}{json}
{   
    "id" : 1234,  
    "field1": "hallo",  
    "field2" : "welt"
}
\end{minted}

Folgender Code-Block taucht im Code-Verzeichnis auf:
\begin{code}{json}{Eine tolle Unterschrift von einem JSON-Codeblock}{code:json-label}
{   
    "id" : 1234,
    "field1": "hallo",
    "field2" : "welt"
}
\end{code}

Codeblöcke können mit \mintinline{latex}{\ref{<labelname>}} referenziert werden, ungefähr so: \ref{code:json-label}.

%% ---------------------------------------------
%% Referenzen
%% ---------------------------------------------
\section{Referenzen}\label{sec_referenzen}
Der Name einer Referenz wird mit \mintinline{LaTeX}{\label{name_der_referenz}} festgelegt. Für einen Abschnitt zum Beispiel: \mintinline{LaTeX}{\section{Referenzen}\label{sec_referenzen}}.

Referenzen auf Bilder, Gleichungen, Kapitel oder andere Elemente können mit \mintinline{LaTeX}{\ref{name_der_referenz}} eingefügt werden. \\
\textbf{Zum Beispiel:} Abbildung \ref{fig_example_pdf}, Abschnitt \ref{sec_referenzen} \\
\textbf{Quellcode für Beispiel:} \mintinline{LaTeX}{Abbildung \ref{fig_example_pdf}, Abschnitt \ref{sec_referenzen}}

Um automatisch das Präfix \glqq Abbildung\grqq{}, \glqq Gleichung\grqq{} oder \glqq  Abschnitt\grqq{} zu erhalten, kann der Befehl \mintinline{LaTeX}{\autoref{name_der_referenz}} genutzt werden. \\
\textbf{Zum Beispiel:} \autoref{fig_example_pdf}, \autoref{sec_referenzen} \\
\textbf{Quellcode für Beispiel:} \mintinline{LaTeX}{\autoref{fig_example_pdf}, \autoref{sec_referenzen}}

%% ---------------------------------------------
%% Zitieren
%% ---------------------------------------------
\section{Zitieren}
Wie durch Quelle \cite{Nobody05} oder \cite{Nobody06} belegt ist...

Angabe von Seitenzahl für die Quelle \cite[S. 5ff]{Nobody07}


%% Anhang
\clearpage
\appendix
\addcontentsline{toc}{chapter}{Anhang} 
\chapter*{Anhang}


%% Literaturverzeichnis
\printbibliography

\end{document}
