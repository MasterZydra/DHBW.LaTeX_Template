\chapter{Beispiele}

%% ---------------------------------------------
%% Abbildung
%% ---------------------------------------------
\section{Abbildung}
In der Abbildung \ref{fig_example} ...

\begin{figure}[!h]
    \centering
    \includegraphics[width=\textwidth]{img/example_image.png}
    \caption{Ein wundervolles Bild in ganzer Breite}
    \label{fig_example}
\end{figure}

Die Abbildung \ref{fig_example_halfsize} hat die 50 Prozent Breite ...

\begin{figure}[h!]
    \centering
    \includegraphics[width=0.5\textwidth]{img/example_image.png}
    \caption[Kurzbeschreibung für Abbildungsverzeichnis]{Ein wundervolles Bild in 50\% Breite}
    \label{fig_example_halfsize}
\end{figure}

%% ---------------------------------------------
%% Abkürzung
%% ---------------------------------------------
\section{Abkürzung}
Eine Verlinkung zur \acs{Abkürzung} ist auch möglich. Jedoch gibt es bei Umlauten Probleme...

%% ---------------------------------------------
%% Anführungszeichen
%% ---------------------------------------------
\section{Anführungszeichen}
\glqq Anführungszeichen\grqq{} können mit \mintinline{latex}{\glqq} für das linke und \mintinline{latex}{\grqq{}} für das rechte Anführungszeichen ergänzt werden.

%% ---------------------------------------------
%% Kurze Tabelle
%% ---------------------------------------------
\section{Kurze Tabelle}
In der Tabelle \ref{tbl_example_table} ...

\begin{table}[!h]
    \centering
    \begin{tabular}{l|ll}
        Datensatz 1     & Wert 1    & Einheit \\
        Datensatz 2     & Wert 2    & Einheit \\ \hline
        $\sum$          & Summe     & Einheit \\
    \end{tabular}
    \caption{Beispiel für eine Tabelle}
    \label{tbl_example_table}
\end{table}

%% ---------------------------------------------
%% Lange Tabelle
%% ---------------------------------------------
\section{Lange Tabelle}
Eine lange Tabelle mit einem alternativen Text für das Tabellenverzeichnis...
\begin{longtable}{p{0.5\linewidth} p{0.5\linewidth}}
    \hline
    Lorem ipsum dolor sit amet, consetetur sadipscing elitr, sed diam nonumy eirmod tempor invidunt ut labore et dolore magna aliquyam erat, sed diam voluptua. &
    Lorem ipsum dolor sit amet, consetetur sadipscing elitr, sed diam nonumy eirmod tempor invidunt ut labore et dolore magna aliquyam erat, sed diam voluptua. \\
    \hline
    Lorem ipsum dolor sit amet, consetetur sadipscing elitr, sed diam nonumy eirmod tempor invidunt ut labore et dolore magna aliquyam erat, sed diam voluptua. &
    Lorem ipsum dolor sit amet, consetetur sadipscing elitr, sed diam nonumy eirmod tempor invidunt ut labore et dolore magna aliquyam erat, sed diam voluptua. \\
    \hline
    Lorem ipsum dolor sit amet, consetetur sadipscing elitr, sed diam nonumy eirmod tempor invidunt ut labore et dolore magna aliquyam erat, sed diam voluptua. &
    Lorem ipsum dolor sit amet, consetetur sadipscing elitr, sed diam nonumy eirmod tempor invidunt ut labore et dolore magna aliquyam erat, sed diam voluptua. \\
    \hline
\caption[Alternativer kürzerer Text für eine Caption]{Hier eine lange Tabelle mit zwei Spalten (je 50\% Breite). Diese könnte auch über mehrere Seiten gehen}
\label{tbl_longtable}
\end{longtable}

%% ---------------------------------------------
%% Quellcode
%% ---------------------------------------------
\section{Quellcode}
Quellcode kann als Block oder im Fließtext stehen. Der Ausdruck \mintinline{python}{print(x**2)} ist ein Beispiel für Inline-Quelltext.

\begin{minted}{json}
{   
    "id" : 1234,  
    "field1": "hallo",  
    "field2" : "welt"
}
\end{minted}

%% ---------------------------------------------
%% Zitieren
%% ---------------------------------------------
\section{Zitieren}
Wie durch Quelle \cite{Nobody06} belegt ist...

Angabe von Seitenzahl für die Quelle \cite[S. 5ff]{Nobody07}
